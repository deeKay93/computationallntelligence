\section{Fazit}\label{sec:fazit}
Da der Prototyp plausible Ergebnisse liefert,
untermauert dies auch die Plausibilität des vorgestellten Konzeptes.
Bei wiederholter Ausführung des Prototyps mit gleichen Parametern ist jedoch festzustellen,
dass gelegentlich ein \enquote{weniger fittes} Ergebnis zurückgegeben wird.
Dies liegt daran, dass die Population in einem lokalen Optimum feststeckt.

Zur Verbesserung sollte eine Strategie entwickelt werden,
um diesem lokalen Optimum zu entkommen.

Sollte der Algorithmus in der Praxis Anwendung finden,
so sollte er zudem mit realen Daten arbeiten, welche aus einer zentralen Quelle geladen werden.

Außerdem beschränkt sich die Fitnessfunktion durch die Normierung darauf,
dass die eingebenden Daten lediglich einen Bereich von einem Jahr abdecken.
Eine mögliche Verbesserung wäre eine dynamischere Normierung.

Insgesamt zeigen dieser Programmentwurf und insbesondere der Prototyp,
dass es möglich ist das Besuchen von Blutspendeterminen mit einem evolutionären Algorithmus zu optimieren.