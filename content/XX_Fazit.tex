\section{Fazit}\label{sec:fazit}
Dass der Prototyp plausible Ergebnisse liefert untermauert die vorgestellten Konzepte.
Bei wiederholer Ausführung des Prototyps mit gleichen Parametern ist jedoch festzustellen,
dass gelegentlich ein \enquote{weniger fittes} Ergebnis zurückgegeben wird.
Dies liegt daran, dass die Population in einem lokalen Optimum feststecken.

Zur Verbesserung sollte eine Stratgie entwickelt werden diesem lokalen Optimum zu entkommen.

Sollte der Algorithmus in der Praxis Anwendung finden,
so sollte er zudem mit reelen Daten arbeiten, welche aus einer zentralen Quelle geladen werden.

Außerdem beschänkt sich die Fitnessfunktion durch die Normierung darauf,
dass die eingebenen Daten sich auf ein Jahr beschränken.
Eine mögliche Verbesserung wäre es hierbei die Normierung dynamischer zu gestalten.

Insgesamt zeigen dieser Programmentwurf und insbesondere der Prototyp,
dass es möglich ist das Besuchen von Blutspendeterminen zu optimieren.