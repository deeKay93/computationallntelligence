\section{Einleitung}\label{sec:einleitung}
Ohne Bltuspenden wäre die moderne Medizin undenkbar.
Sei es für die Behandlung von Krebs- und Herzerkrankungen,
für die Unfallchirugie oder für andere medizinische Eingriffe,
täglich werden deutschlandweit circa 15.000 Bltuspenden benötigt. \cite{FragenAn98:online}

Unter anderem bietet das Deutsche Rote Kreuz regelmäßig in verschiedenen Orten und Gemeinden die Möglichkeit Blut zu spenden.

Aus gesundheitlichen Gründen ist es jedoch nur unter bestimmten Bedingungen gestattet Blut zu spenden.
Hierzu zählen unter anderem \cite{SpendeCh65:online}:
\begin{itemize}
    \setlength\itemsep{-0.5em}
    \item Alter zwischen 18 und 65 Jahre
    \item Mindestgewicht 50 kg
    \item Keine Erkältung, o.Ä.
    \item Kein Arzttermin innerhalb der letzen 7 Tage
    \item Keine Opertationen innerhalb von 12 Monaten
\end{itemize}

Bei diesen Bedingungen hat der anwesende Arzt einen gewissen Ermessenspielraum und kann eine Spende dennoch genehmigen.
Es gibt jedoch auch \enquote{harte} Bedingungen, welche eine Spende ausschließen können:

\begin{itemize}
    \setlength\itemsep{-0.5em}
    \item Keine Spende innerhalb der letzten 56 Tage
    \item Innerhalb von 12 Monaten: \begin{itemize}
        \vspace{-1em}
        \setlength\itemsep{-0.5em}
        \item Männer:  maximal 6 Spenden
        \item Frauen:  maximal 4 Spenden
    \end{itemize}
\end{itemize}

In diesem Programmentwurf wird erläutert,
wie ein evolutionärer Algorithmus genutzt werden kann,
um möglichst oft zu Spenden und
gleichzeitig eine möglichst geringe Gesamtstecke zwischen einem Ausgangsort und den Spendeorten zurückzulegen.
