\section{Prototyp}\label{sec:prototyp}

Der diesem Programmentwurf angehängte Prototyp basiert grundlegend auf der Implementierung in \cite{Quiz15Th91:online}.
Diese Implementierung löst das Knappsackproblem mit Hilfe eines Genitischen Algorithmus.
Sie wurde auf die hier beschriebenen Methoden angepasst und unter anderm
in folgenden Punkten geändert:
\begin{itemize}
    \item Eingabe der Daten über eine wohlgeformte CSV-Datei anstatt einer unübersichtlichen Textdatei.
    \item Entfernen der ungültigen Gene aus der Population anstatt sie \enquote{nur} mit einer Fitness von $0$ zu versehen.
    \item Anpassung der initialen Population
\end{itemize}


\noindent
Zur Validierung des Prototyps stehen Testdaten zur Verfügung.
Die Orte der Tesdtdaten entstammen aus tatsächlichen Spendeorten in einem Umkreis von 25 km um einen bestimmten Ort.
Da lediglich Daten bis Mitte Dezember 2018 vorliegen,
werden für die Testdaten zufällige Termine für das Jahr 2019 erzeugt.


% Echte Daten von:
% https://www.blutspende.de/infos-zur-blutspende/blutspende-termine/blutspendetermine.php?umkreis=25&abgeschickt=1&plz_ort_eingabe=74076&d_v_eingabe=13.09.2018

\subsection{Entwicklung der Fitness}

Führt man den Prototypen mit den Testdaten aus so lässt sich für die Fitnesswerte die Entwicklung in \autoref{fig:fitness} beobachten.
Zwar fällt die schlechteste Fitness der ersten Generation im Vergleich zur initialen,
aber sowohl die durchschnittliche, als auch die beste Fitness ansteigen.
Dies bedeutet, dass zwar mindestens ein Gen mit einer schlechteren Fitness entsteht,
aber insgesamt \enquote{bessere} Gene erzeugt werden.

Der beste und der durchschnittliche Fitnesswert konvertieren gegen $\sim 0,953$.
Dies ist auch das in diesem Durchlauf erreichte Optimum.
Das beste Gen erreicht diesen Wert bereits in der achten Generation.

Mit die schlechteste Fitness kovergiert gegen $\sim 0,83$,
macht aber dann in der letzten Generation einen Sprung zu $\sim 0,953$.
Wahrscheinlich werden in dieser Generation die letzen \enquote{weniger guten} Gene durch das beste gefundene Gen verdrängt.

Interessanterweise sind in der vierzehnten Generation
die Fitnesswerte des besten und des schlechtesten Gens identisch mit $0,9531284454244\highlight{lime!30}{763}$.
Der Gesamtdurchschnitt wird jedoch lediglich mit $0,9531284454244\highlight{lime!30}{614}$ angegeben.
In den letzten drei Ziffern ist der Durchschnitt also geringer als das Minimum.
Mathematisch ist dies zwar unmöglich, kann aber auf numerische Fehler bei der Berechnug zurückgeführt werden.

\begin{figure}[ht]
    \centering
    \begin{tikzpicture}
        \begin{axis}[
            reverse legend,
            xlabel={Generation},
            ylabel={Fitness},
            legend pos=south east,
            width=12cm,
            height=7.41cm,
            enlarge x limits=false,
            enlarge y limits=false,
            ymin=0,
            ymax=1
        ]
        \addplot[smooth,mark=*,LineRed] plot coordinates {
            (0,0.2870406100698273)
            (1,0.2099411980889379)
            (2,0.31006982726938626)
            (3,0.39900312385152514)
            (4,0.4881201764057332)
            (5,0.5721609702315326)
            (6,0.6576189819919147)
            (7,0.6866432377802278)
            (8,0.7322882212421904)
            (9,0.7915803013597943)
            (10,0.8064645350973907)
            (11,0.8143697170158031)
            (12,0.8251690554943036)
            (13,0.8288239617787578)
            (14,0.9531284454244763)
        };
        \addlegendentry{Schlechteste Fitness}
        \addplot[smooth,mark=diamond*,LineBlue] plot coordinates {
            (0,0.487608275473627)
            (1,0.5212174183065748)
            (2,0.5873159147484383)
            (3,0.6612959575750835)
            (4,0.7191537766318545)
            (5,0.780470746258059)
            (6,0.8203987050017139)
            (7,0.8696678703445766)
            (8,0.9067217993824239)
            (9,0.920546737735418)
            (10,0.9326826212355643)
            (11,0.941074863523486)
            (12,0.9465031836095645)
            (13,0.952438020104876)
            (14,0.9531284454244614)
        };
        \addlegendentry{Durchschnittliche Fitness}
        \addplot[smooth,mark=triangle*,LineGreen] plot coordinates {
            (0,0.5920066152149945)
            (1,0.7430742374127159)
            (2,0.8150973906651967)
            (3,0.9110529217199559)
            (4,0.9130374862183022)
            (5,0.9314773980154355)
            (6,0.9410832414553473)
            (7,0.9429162072767365)
            (8,0.9531284454244763)
            (9,0.9531284454244763)
            (10,0.9531284454244763)
            (11,0.9531284454244763)
            (12,0.9531284454244763)
            (13,0.9531284454244763)
            (14,0.9531284454244763)
        };
        \addlegendentry{Beste Fitness}
        \end{axis}
    \end{tikzpicture}
	\caption{Fitnesswerte der verschiedenen Generationen}
	\label{fig:fitness}
\end{figure}


\subsection{Entwicklung gültiger Gene}

\autoref{fig:ueberlebende} zeigt die Anzahl der gültigen Gene im Verlauf der Generationen.
In der initialen Generation erfüllen lediglich $338$ Gene - etwa ein Drittel - die gestellten Bedingungen.
Von Generation $1$ bis $9$ schwank diese Zahl zwischen $848$ und $916$ um dann zu steigen,
 bis in Generation $14$ alle Gene gültig sind.

\begin{figure}[ht]
    \centering
    \begin{tikzpicture}
        \begin{axis}[
            legend pos=south east,
            enlarge x limits=false,
            enlarge y limits=false,
            width=12cm,
            height=7.41cm,
            ymin=300,
            ymax=1000
        ]
        \addplot[
            name path=B,
            LineGreen,
            mark=*
            ] coordinates {
            (0, 338)
            (1, 916)
            (2, 908)
            (3, 883)
            (4, 848)
            (5, 866)
            (6, 866)
            (7, 872)
            (8, 865)
            (9, 877)
            (10,951)
            (11,988)
            (12,990)
            (13,999)
            (14,1000)
        } \closedcycle;
        \end{axis}
    \end{tikzpicture}
	\caption{Anzahl gültiger Gene der verschiedenen Generationen}
	\label{fig:ueberlebende}
\end{figure}


\todoask[inline]{Entwicklung 1/5-Regel?}

\subsection{Bewertung}
Daraus, dass in der letzten Generation sowohl Minimum und Maximum die gleiche Fitness haben,
als auch alle Gene die Bedingungen erfüllen kann geschlossen werden, dass in dieser Generation alle Gene identisch sind.
Somit endet der Algorithmus sinnvolerweiße an dieser Stelle.

