\section{Prototyp}\label{sec:prototyp}

Der diesem Programmentwurf angehängte Prototyp basiert grundlegend auf der Implementierung in \cite{Quiz15Th91:online}.
Diese Implementierung löst das Knappsackproblem mit Hilfe eines Genitischen Algorithmus.
Sie wurde auf die hier beschriebenen Methoden angepasst und unter anderm
in diesen Punkten geändert:
\begin{itemize}
    \item Eingabe der Daten über eine wohlgeformte CSV-Datei anstatt einer unübersichtlichen Textdatei.
    \item Entfernen der ungültigen Gene aus der Population anstatt sie \enquote{nur} mit einer Fitness von $0$ zu versehen.
    \item Anpassung der initialen Population
\end{itemize}


\noindent
Zur Validierung des Prototyps stehen Testdaten zur Verfügung.
Die Orte der Tesdtdaten entstammen aus tatsächlichen Spendeorten in einem Umkreis von 25 km um einen bestimmten Ort.
Da lediglich Daten bis Mitte Dezember 2018 vorliegen,
werden für die Testdaten zufällige Termine für das Jahr 2019 erzeugt.

In den nächsten Unterabschnitten wird der Verlauf verscheidener Werte während eines Durchlaufes betrachtet.
Hierzu wird das Programm mit folgenden Parametern ausgeführt:
\begin{table}[ht]% Try here, and then top
    \begin{center}
        \begin{tabular}{l|c}
            Parameter                       & Wert \\
            \hline
            Population                      & $500$  \\
            Maximale Iterationen            & $1000$ \\
            Anteil der Elite                & $20\%$ \\
            Mutationswahrscheinlichkeit     & $80\%$ \\
            Initiale Schrittweite           & $3$     \\
            Präzision für Stoppbedingung  & $10^{-4}$
        \end{tabular}
    \end{center}
    \caption{Parameter }
  \end{table}


% Echte Daten von:
% https://www.blutspende.de/infos-zur-blutspende/blutspende-termine/blutspendetermine.php?umkreis=25&abgeschickt=1&plz_ort_eingabe=74076&d_v_eingabe=13.09.2018

\subsection{Entwicklung der Fitness}

Führt man den Prototypen mit den Testdaten aus so lässt sich für die Fitnesswerte die Entwicklung in \autoref{fig:fitness} beobachten.
Zwar fällt die schlechteste Fitness der ersten Generation im Vergleich zur initialen,
aber sowohl die durchschnittliche, als auch die beste Fitness ansteigen.
Dies bedeutet, dass zwar mindestens ein Gen mit einer schlechteren Fitness entsteht,
jedoch insgesamt \enquote{bessere} Gene erzeugt werden.

Der beste und der durchschnittliche Fitnesswert konvertieren gegen ca. $0,953$.
Dies ist auch das in diesem Durchlauf erreichte Optimum.
Das beste Gen erreicht diesen Wert bereits in der siebten Generation.

Die schlechteste Fitness fällt nach der initialen Generation ab,
steigt dann aber wieder an und konvergiert zunächst gegen ca. $0,82$,
macht jedoch dann in der letzten Generation einen Sprung zu ca. $0,953$.
Wahrscheinlich werden in dieser Generation die verbleibenden \enquote{weniger guten} Gene durch das beste gefundene Gen verdrängt.

Interessanterweise sind in der zwölften Generation
die Fitnesswerte des besten und des schlechtesten Gens identisch mit $0,9531284454244\highlight{lime!30}{763}$.
Der Gesamtdurchschnitt wird jedoch lediglich mit $0,9531284454244\highlight{lime!30}{685}$ angegeben.
In den letzten drei Ziffern ist der Durchschnitt also geringer als das Minimum.
Mathematisch ist dies zwar unmöglich, kann aber auf numerische Fehler bei der Berechnug zurückgeführt werden.

\begin{figure}[ht]
    \centering
    \begin{tikzpicture}
        \begin{axis}[
            reverse legend,
            xlabel={Generation},
            ylabel={Fitness},
            legend pos=south east,
            width=12cm,
            height=7.41cm,
            enlarge x limits=false,
            enlarge y limits=false,
            ymin=0,
            ymax=1
        ]
        \addplot[mark=*,LineRed] plot coordinates {
            (0,0.39809812568908487)
            (1,0.3137908857037854)
            (2,0.35191106210951856)
            (3,0.43761944138184494)
            (4,0.4597023153252481)
            (5,0.5794377067254686)
            (6,0.6075523704520397)
            (7,0.697827085630283)
            (8,0.7221536199926498)
            (9,0.8083664094083058)
            (10,0.8288239617787578)
            (11,0.8288239617787578)
            (12,0.9531284454244763)
        };
        \addlegendentry{Schlechteste Fitness}
        \addplot[mark=diamond*,LineBlue] plot coordinates {
            (0,0.5010904812526614)
            (1,0.533322685651329)
            (2,0.5815548190310615)
            (3,0.6395867099505065)
            (4,0.700842131783521)
            (5,0.7559416493970084)
            (6,0.8170118889289913)
            (7,0.870926735946039)
            (8,0.9198439253202729)
            (9,0.9420204083027279)
            (10,0.9454525511769829)
            (11,0.9520316362842004)
            (12,0.9531284454244685)
        };
        \addlegendentry{Durchschnittliche Fitness}
        \addplot[mark=triangle*,LineGreen] plot coordinates {
            (0,0.6075248070562294)
            (1,0.7140963800073502)
            (2,0.7776552737963984)
            (3,0.8356045571481074)
            (4,0.8917998897464168)
            (5,0.9459619625137817)
            (6,0.9459619625137817)
            (7,0.9531284454244763)
            (8,0.9531284454244763)
            (9,0.9531284454244763)
            (10,0.9531284454244763)
            (11,0.9531284454244763)
            (12,0.9531284454244763)
        };
        \addlegendentry{Beste Fitness}
        \end{axis}
    \end{tikzpicture}
	\caption{Fitnesswerte der verschiedenen Generationen}
	\label{fig:fitness}
\end{figure}

\newpage
\subsection{Entwicklung gültiger Gene}

\autoref{fig:ueberlebende} zeigt die Anzahl der gültigen Gene im Verlauf der Generationen.
In der initialen Generation erfüllen lediglich $338$ Gene - etwa ein Drittel - die gestellten Bedingungen.
Von Generation $1$ bis $9$ schwank diese Zahl zwischen $848$ und $916$ um dann zu steigen,
 bis in Generation $14$ alle Gene gültig sind.

\begin{figure}[ht]
    \centering
    \begin{tikzpicture}
        \begin{axis}[
            legend pos=south east,
            enlarge x limits=false,
            enlarge y limits=false,
            width=8.09cm,
            height=5cm,
            ymin=375,
            ymax=500
        ]
        \addplot[
            name path=B,
            LineGreen,
            mark=*
            ] coordinates {
            (0, 391)
            (1, 461)
            (2, 449)
            (3, 454)
            (4, 446)
            (5, 432)
            (6, 446)
            (7, 436)
            (8, 452)
            (9, 484)
            (10,485)
            (11,498)
            (12,500)
        };
        \end{axis}
    \end{tikzpicture}
	\caption{Anzahl gültiger Gene}
	\label{fig:ueberlebende}
\end{figure}

\subsection{Entwicklung der Mutationsrate}

In \autoref{fig:mutationsrate} wird die Entwicklung der Mutationsrate dargestellt.
Der Anstieg in den ersten beiden Generationen und späeter Rückgang zeigt,
dass gerade am Anfang durch Mutationen eine deutliche Verbesserung der Gene erreicht werden kann.
Gegen Ende nimmt diese Verbesserung ab. Hieraus lässt sich auf ein (lokales) Optimum schließen.

\begin{figure}[ht]
    \centering
    \begin{tikzpicture}
        \begin{axis}[
            legend pos=south east,
            enlarge x limits=false,
            enlarge y limits=false,
            width=8.09cm,
            height=5cm,
            ymin=0,
            ymax=5
        ]
        \addplot[
            LineBlue,
            mark=*
            ] coordinates {
            (0, 3)
            (1, 4)
            (2, 5)
            (3, 4)
            (4, 3)
            (5, 2)
            (6, 1)
            (7, 1)
            (8, 1)
            (9, 1)
            (10,1)
            (11,1)
            (12,1)
        };
        \end{axis}
    \end{tikzpicture}
    \caption{Mutationsrate}
    \label{fig:mutationsrate}
\end{figure}

