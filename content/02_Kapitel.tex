\section{Evolutionärer Algorithmus}\label{sec:evol-alg}
\todoask[inline]{Verlgeich Knappsack?}
Für den Algorithmus wird angenommen,
dass die eine Liste der möglichen Spendetermine wie folgt eingegeben wird:
\begin{center}
    \begin{tabular}{l|c|c|c|c}
        PLZ     & Ort               &  Straße               & Datum     & Distanz   \\
        \hline
        12345   & Musterstadt       & Hauptstraße 42        & 5.3.2019  & 6.31      \\
        04711   & Beispieldorf      & Spenderstraße. 13     & 894.2019  & 20.42     \\
    \end{tabular}
\end{center}

Für den Algorithmus selbst sind hierbei lediglich die letzen beiden Merkmale notwendig.
Die anderen Merkmale dienen der finalen Darstellung für den Anwender.

\paragraph{Bewertungsfunktion}
Die Bewertungsfunktion hat das Ziel die Anzahl der besuchten Termine zu maximieren und die Gesamtstrecke zu minimieren.
Dies soll unter Beachtung der bereits vorgestellten Bedingungen geschehen.

Im Vergleich zu beispielsweiße einer Produktionsplanung gibt es bei den Bedingungen keinen Spielraum.
In einer Produktionsplanung ist es möglicherweiße gestattet die maximale Anzahl der verplanten Stunden zu überschreiten.
Dies wird in der Fitness entsprechend bestraft. Aber sollte das beste Ergebnis Überstunden enthalten ist es dennoch ein gültiges Ergebnis.

Die Bedingungen für das Blutspenden sind jedoch \enquote{harte} Bedingungen.
Das heißt, dass beispielsweiße keine Spende möglich ist, wenn die letze Spende erst 55 Tage her ist und nicht die geforderten 56.
Um solche Ergebnisse zu verhindern ist de Fitness für diese Gene $0$.

Insgesamt lässt sich die Fitness eines Gens nach Formel \ref{eqn:fitness} berechnen.

\begin{equation}
    \label{eqn:fitness}
    f(x)=
    \begin{cases}
        0,              & \text{wenn Tage zwischen zwei Terminen} < 56 \\
        0,              & \text{wenn Anzahl Spenden in 12 Monaten} > 6 \\
        w_c * \sum x_i + (1 - w_c) * \sum x_i * d_i & \text{sonst}
    \end{cases}
\end{equation}


\paragraph{Genstring}
Der Genstring wird als Bitstring modelliert.
Eine $0$ steht für Termine, welche nicht wahrgenommen werden und eine $1$ ensprechend für besuchte Termine.


\paragraph{Mutationen}

\newpage

\todo[inline]{Spezifikation einer Bewertungsfunktion,
um die Verbesserung von diesen Entscheidungen messen zu können.}

\todo[inline]{Für lokale Verbesserungsverfahren bzw. Evolution}

\todo[inline]{Repräsentation der Lösung als Genstring: binär,
ganzzahlig, reelle Zahlen, Permution}
\todo[inline]{Zugehörige Mutationen: Bitinversion, Änderung der Zahl,
Änderung normalverteilt, (Swap, Insert, Teilstringinversion)}
\todo[inline]{Zugehörige Rekombination: 1-point-crossover, ??,
arithmetische Rekombination, (order crossover OX, partial map crossover PMX)}
\todo[inline]{wichtige Maßnahmen zur Effizienz
- Nachbarschaft (Austausch „ähnlicher Gene“)
- Evolution auf lokalen Minima
- Schrittweitensteuerung durch 1/5-Regel oder Erweiterung Genstring}

